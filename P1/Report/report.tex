\documentclass{report}

% Language setting
\usepackage[main=portuguese, english]{babel}
\usepackage{csquotes}

% Set page size and margins
\usepackage[a4paper,top=2cm,bottom=2cm,left=3cm,right=3cm,marginparwidth=1.5cm]{geometry}

% Useful packages
\usepackage{ulem}
\usepackage{parskip}
\usepackage{indentfirst}
\usepackage{setspace}
\usepackage{amsmath}
\usepackage{relsize}
\usepackage{array}

\usepackage{graphicx}
\usepackage{xcolor}
\usepackage{colortbl}
\usepackage{subfigure}
\usepackage{titlesec}
\usepackage[colorlinks=false, allbordercolors={0 0 0}, pdfborderstyle={/S/U/W 0.25}]{hyperref}
\usepackage[hypcap=true]{caption}
\usepackage{enumitem}
\usepackage{soul}

\usepackage{tikz}
\usepackage{tikz-3dplot}

% Set section numbering from 1.1
\renewcommand{\thesection}{\arabic{section}.1}

\let\oldsection\section
\renewcommand\section{\clearpage\oldsection}

% Change section formatting
\titleformat{\section}
  {\fontsize{12}{15}\selectfont\bfseries}{\thesection}{1em}{}

% Configure indentations
\setlength{\parindent}{1.5cm}

\begin{document}

    \begin{titlepage}
        \centering
        
        \LARGE {Universidade Federal do Rio Grande do Sul \\ Escola de Engenharia}
    
        \begin{figure}[h!]
        \centering
        \subfigure
        {\includegraphics[width=0.35\linewidth]{images/logos/UFRGS.png}}
        \hspace{1cm}
        \subfigure
        {\includegraphics[width=0.3\linewidth]{images/logos/EE.png}}
        \end{figure}
    
        \LARGE {ENG04010 \\ Teoria Eletromagnética e Ondas}
        
        \vfill
        {\noindent\hrulefill \\
        \bfseries \Huge{Trabalho Complementar} \\ \LARGE{Resolução de Problemas de Valor de Contorno} \\
        \noindent\hrulefill}
        
        \vfill
        {\LARGE Pedro Lubaszewski Lima (00341810) \\~\\ Turma U}
    
        \vfill
        {\LARGE 23 de dezembro de 2024}
        
    \end{titlepage}

        \renewcommand{\contentsname}{Sumário}
        \tableofcontents
        \clearpage
        \addtocontents{toc}{\protect\thispagestyle{empty}}

\section{Enunciado do Problema}

Com o intuito de exercitar os conhecimentos ensinados sobre Problemas de Valores de Contorno (PVC)
em Eletrostática, foi proposto o seguinte exercício a ser resolvido:

Considere um cubo oco de dimensões laterais $ a $, composto de faces condutoras ideais, conforme
a figura abaixo. Suponha que exista uma pequena separação entre cada face. As faces laterais,
em tom mais claro, são mantidas em um potêncial nulo. A face superior ($ 0 < x < a $, $ 0 < y < a $,
$ z = a $) é mantida em potencial contante e uniforme $ V_0 $.

\begin{figure}[h!]
    \centering
    \tdplotsetmaincoords{60}{130}
    \begin{tikzpicture}[tdplot_main_coords]
        % Define the cube's dimensions
        \def\a{2} % length of the cube's sides
        \def\d{4} % length of the axis

        % Draw the axis
        \draw[thick,->] (0,0,0)-- (\d,0,0) node[anchor=north east]{$x$};
        \draw[thick,->] (0,0,0)-- (0,\d,0) node[anchor=north west]{$y$};
        \draw[thick,->] (0,0,0)-- (0,0,\d) node[anchor=south]{$z$};

        % Draw the cube
        \filldraw[fill=black!10, draw=black] (0,0,0) -- (\a,0,0) -- (\a,\a,0) -- (0,\a,0) -- cycle;
        \filldraw[fill=black!10, draw=black] (0,0,0) -- (0,\a,0) -- (0,\a,\a) -- (0,0,\a) -- cycle;
        \filldraw[fill=black!10, draw=black] (0,0,0) -- (0,0,\a) -- (\a,0,\a) -- (\a,0,0) -- cycle;
        \filldraw[fill=black!10, draw=black] (\a,0,0) -- (\a,\a,0) -- (\a,\a,\a) -- (\a,0,\a) -- cycle;
        \filldraw[fill=black!10, draw=black] (0,\a,0) -- (\a,\a,0) -- (\a,\a,\a) -- (0,\a,\a) -- cycle;
        \filldraw[fill=black!30, draw=black] (0,0,\a) -- (\a,0,\a) -- (\a,\a,\a) -- (0,\a,\a) -- cycle;
    
        % Label the cube's faces and points on the axis
        \node at (\a,0.5*\a,0.5*\a) {$ V = 0 $};
        \node at (0.5*\a,\a,0.5*\a) {$ V = 0 $};
        \node at (0.5*\a,0.5*\a,\a) {$V = V_0 $};

        \node at (\a+0.25,0,0.25) {$ a $};
        \node at (0,\a+0.25,0.25) {$ a $};
        \node at (0,0.25,\a+0.25) {$ a $};
    \end{tikzpicture}
    \caption{\label{plot:assignment} Cubo Condutor de Dimensões Laterais $ a $.}
\end{figure}

Com isso em mente, faça o que se pede:
\begin{enumerate}
    \item Determine uma equação para o potencial no interior do cubo de forma analítica, utilizando o Método da Separação de Variáveis.
    \item Esboce o potencial, na forma de um ``mapa de calor”, para a região central do cubo (fixando $ x = \frac{a}{2} $
    ou $ y = \frac{a}{2} $ e variando as outras duas variáveis), utilizando resultados obtidos numericamente.
\end{enumerate}

\section{Resolução Analítica do Problema}

Para modelar o comportamento de $ V(x, y, z) $ analiticamente, partir-se-á da Equação de Laplace com as condições de contorno fornecidas
no problema:

\begin{equation}
  \label{eq:laplace}
  \centering
  \begin{cases}
    \nabla^2 V = 0 \\
    x: V(0, y, z) = 0,\, V(a, y, z) = 0 \\
    y: V(x, 0, z) = 0,\, V(x, a, z) = 0 \\
    z: V(x, y, 0) = 0,\, V(x, y, a) = V_0
  \end{cases}
\end{equation}

A partir dela, tem-se, em coordenadas cartesianas, que:
$$ \frac{\partial^2 V}{\partial x^2} + \frac{\partial^2 V}{\partial y^2} + \frac{\partial^2 V}{\partial z^2} = 0 $$

E, pelo Método da Separação de Variáveis, assume-se que, para coordenadas cartesianas:
$$ V(x, y, z) = X(x)Y(y)Z(z) $$

Portanto, a partir daí, tem-se que:
$$ \Rightarrow \frac{\partial^2 V}{\partial x^2} = Y(y)Z(z)\frac{\partial^2}{\partial x^2}X(x) $$
$$ \Rightarrow \frac{\partial^2 V}{\partial y^2} = X(x)Z(z)\frac{\partial^2}{\partial y^2}Y(y) $$
$$ \Rightarrow \frac{\partial^2 V}{\partial z^2} = X(x)Y(y)\frac{\partial^2}{\partial z^2}Z(z) $$
$$ \Rightarrow \nabla^2 V = Y(y)Z(z)\frac{\partial^2}{\partial x^2}X(x) + X(x)Z(z)\frac{\partial^2}{\partial y^2}Y(y) + X(x)Y(y)\frac{\partial^2}{\partial z^2}Z(z) = 0 $$

Agora, assumindo que $ X(x) \neq 0 $, $ Y(y) \neq 0 $ e $ Z(z) \neq 0 $, na região de interesse, pode-se dividir a equação acima por $ X(x)Y(y)Z(z) $:
$$ \Rightarrow \frac{1}{X(x)}\frac{\partial^2 X(x)}{\partial x^2} + \frac{1}{Y(y)}\frac{\partial^2 Y(y)}{\partial y^2} + \frac{1}{Z(z)}\frac{\partial^2 Z(z)}{\partial z^2} = 0 $$

Porém, a única forma dessa equação resultar em zero para todos os valores de $ X(x) $, $ Y(y) $ e $ Z(z) $ se dá quando cada uma das parcelas somadas na equação é uma constante.
Em outras palavras:
\begin{equation}
  \label{eq:laplace_constants}
  \centering
  \begin{cases}
    \frac{1}{X(x)}\frac{\partial^2 X(x)}{\partial x^2} = -K_x^2 \\
    \frac{1}{Y(y)}\frac{\partial^2 Y(y)}{\partial y^2} = -K_y^2 \\
    \frac{1}{Z(z)}\frac{\partial^2 Z(z)}{\partial z^2} = -K_z^2
  \end{cases}
  \Rightarrow K_x^2 + K_y^2 + K_z^2 = 0
\end{equation}

Essa escolha de constantes foi feita para facilitar a dedução do resto do problema, visto que as constantes podem ser complexas.

Multiplicando cada uma das equações de \ref{eq:laplace_constants} pelas suas respectivas funções dependentes apenas de uma coordenada e somando
a constante dos dois lados das equações obtém-se o seguinte sistema de Equações Diferenciais Ordinárias (EDOs):
\begin{equation}
    \label{eq:laplace_edos}
    \centering
    \begin{cases}
      \frac{d^2 X(x)}{dx^2} + X(x)K_x^2 = 0 \\
      \frac{d^2 Y(y)}{dy^2} + Y(y)K_y^2 = 0 \\
      \frac{d^2 Z(z)}{dz^2} + Z(z)K_z^2 = 0 \\
      K_x^2 + K_y^2 + K_z^2 = 0
    \end{cases}
  \end{equation}

\section{Resolução Numérica do Problema}
\subsection{Comportamento Numérico da Solução}
\subsection{Simulação em \textit{Software}}

\section{Exemplos}

\begin{itemize}
  \item $ N_1 = 3$;
  \item $ N_2 = 4$;
  \item $ N_3 = 1$;
  \item $ N_4 = 8$;
  \item $ N_5 = 1$;
  \item $ N_6 = 0$.
\end{itemize}

\begin{center}
  \[ \scalebox{3}{$ * $} \]
\end{center}

\begin{equation}
  \label{eq:quad_cascade}
  \centering
  \begin{matrix}
    a_{11} = a^{'}_{11}a^{''}_{11} + a^{'}_{12}a^{''}_{21} & \hspace{10pt} a_{12} = a^{'}_{11}a^{''}_{12} + a^{'}_{12}a^{''}_{22} \\\\
    a_{21} = a^{'}_{21}a^{''}_{11} + a^{'}_{22}a^{''}_{21} & \hspace{10pt} a_{22} = a^{'}_{21}a^{''}_{12} + a^{'}_{22}a^{''}_{22}
  \end{matrix}
\end{equation}

\end{document}