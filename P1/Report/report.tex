\documentclass{report}

% Language setting
\usepackage[main=portuguese, english]{babel}
\usepackage{csquotes}

% Set page size and margins
\usepackage[a4paper,top=2cm,bottom=2cm,left=3cm,right=3cm,marginparwidth=1.5cm]{geometry}

% Useful packages
\usepackage{ulem}
\usepackage{parskip}
\usepackage{indentfirst}
\usepackage{setspace}
\usepackage{amsmath}
\usepackage{amssymb}
\usepackage{relsize}
\usepackage{array}

\usepackage{graphicx}
\usepackage{xcolor}
\usepackage{colortbl}
\usepackage{subfigure}
\usepackage{titlesec}
\usepackage[colorlinks=false, allbordercolors={0 0 0}, pdfborderstyle={/S/U/W 0.25}]{hyperref}
\usepackage[hypcap=true]{caption}
\usepackage{enumitem}
\usepackage{soul}

\usepackage{tikz}
\usepackage{tikz-3dplot}

% Set section numbering from 1.1
\renewcommand{\thesection}{\arabic{section}.1}

\let\oldsection\section
\renewcommand\section{\clearpage\oldsection}

% Change section formatting
\titleformat{\section}
  {\fontsize{12}{15}\selectfont\bfseries}{\thesection}{1em}{}

% Configure indentations
\setlength{\parindent}{1.5cm}

\begin{document}

    \begin{titlepage}
        \centering
        
        \LARGE {Universidade Federal do Rio Grande do Sul \\ Escola de Engenharia}
    
        \begin{figure}[h!]
        \centering
        \subfigure
        {\includegraphics[width=0.35\linewidth]{images/logos/UFRGS.png}}
        \hspace{1cm}
        \subfigure
        {\includegraphics[width=0.3\linewidth]{images/logos/EE.png}}
        \end{figure}
    
        \LARGE {ENG04010 \\ Teoria Eletromagnética e Ondas}
        
        \vfill
        {\noindent\hrulefill \\
        \bfseries \Huge{Trabalho Complementar} \\ \LARGE{Resolução de Problemas de Valor de Contorno} \\
        \noindent\hrulefill}
        
        \vfill
        {\LARGE Pedro Lubaszewski Lima (00341810) \\~\\ Turma U}
    
        \vfill
        {\LARGE 23 de dezembro de 2024}
        
    \end{titlepage}

        \renewcommand{\contentsname}{Sumário}
        \tableofcontents
        \clearpage
        \addtocontents{toc}{\protect\thispagestyle{empty}}

\section{Enunciado do Problema}

Com o intuito de exercitar os conhecimentos ensinados sobre Problemas de Valores de Contorno (PVC)
em Eletrostática, foi proposto o seguinte exercício a ser resolvido:

Considere um cubo oco de dimensões laterais $ a $, composto de faces condutoras ideais, conforme
a figura abaixo. Suponha que exista uma pequena separação entre cada face. As faces laterais,
em tom mais claro, são mantidas em um potêncial nulo. A face superior ($ 0 < x < a $, $ 0 < y < a $,
$ z = a $) é mantida em potencial contante e uniforme $ V_0 $.

\begin{figure}[h!]
    \centering
    \tdplotsetmaincoords{60}{130}
    \begin{tikzpicture}[tdplot_main_coords]
        % Define the cube's dimensions
        \def\a{2} % length of the cube's sides
        \def\d{4} % length of the axis

        % Draw the axis
        \draw[thick,->] (0,0,0)-- (\d,0,0) node[anchor=north east]{$x$};
        \draw[thick,->] (0,0,0)-- (0,\d,0) node[anchor=north west]{$y$};
        \draw[thick,->] (0,0,0)-- (0,0,\d) node[anchor=south]{$z$};

        % Draw the cube
        \filldraw[fill=black!10, draw=black] (0,0,0) -- (\a,0,0) -- (\a,\a,0) -- (0,\a,0) -- cycle;
        \filldraw[fill=black!10, draw=black] (0,0,0) -- (0,\a,0) -- (0,\a,\a) -- (0,0,\a) -- cycle;
        \filldraw[fill=black!10, draw=black] (0,0,0) -- (0,0,\a) -- (\a,0,\a) -- (\a,0,0) -- cycle;
        \filldraw[fill=black!10, draw=black] (\a,0,0) -- (\a,\a,0) -- (\a,\a,\a) -- (\a,0,\a) -- cycle;
        \filldraw[fill=black!10, draw=black] (0,\a,0) -- (\a,\a,0) -- (\a,\a,\a) -- (0,\a,\a) -- cycle;
        \filldraw[fill=black!30, draw=black] (0,0,\a) -- (\a,0,\a) -- (\a,\a,\a) -- (0,\a,\a) -- cycle;
    
        % Label the cube's faces and points on the axis
        \node at (\a,0.5*\a,0.5*\a) {$ V = 0 $};
        \node at (0.5*\a,\a,0.5*\a) {$ V = 0 $};
        \node at (0.5*\a,0.5*\a,\a) {$V = V_0 $};

        \node at (\a+0.25,0,0.25) {$ a $};
        \node at (0,\a+0.25,0.25) {$ a $};
        \node at (0,0.25,\a+0.25) {$ a $};
    \end{tikzpicture}
    \caption{\label{plot:assignment} Cubo Condutor de Dimensões Laterais $ a $.}
\end{figure}

Com isso em mente, faça o que se pede:
\begin{enumerate}
    \item Determine uma equação para o potencial no interior do cubo de forma analítica, utilizando o Método da Separação de Variáveis
    (discutido na Seção \ref{sec:analysis}).
    \item Esboce o potencial, na forma de um ``mapa de calor”, para a região central do cubo (fixando $ x = \frac{a}{2} $
    ou $ y = \frac{a}{2} $ e variando as outras duas variáveis), utilizando resultados obtidos numericamente (discutido na Seção
    \ref{sec:numerical}).
\end{enumerate}

\section{Resolução Analítica do Problema}
\label{sec:analysis}

Partindo de primeiros princípios, utilizando dos postulados da Eletrostática:
$$ \overrightarrow{\nabla}\cdot \overrightarrow{E} = \frac{\rho_V}{\varepsilon_0} $$
$$ \overrightarrow{\nabla}\times \overrightarrow{E} = \overrightarrow{0} $$

Com a segunda expressão, deduz-se que o campo elétrico é conservativo, ou seja,
$$ \Rightarrow \exists V \mid \overrightarrow{E} = -\overrightarrow{\nabla} V $$

Logo, unindo essa equação e a primeira equação dessa seção:
$$ \Rightarrow \overrightarrow{\nabla}\cdot(-\overrightarrow{\nabla} V) = \frac{\rho_V}{\varepsilon_0} $$
$$ \Rightarrow \nabla^2V = -\frac{\rho_V}{\varepsilon_0} \text{ (Equação de Poisson)} $$

Para o caso do problema onde não há cargas onde procura-se determinar o potencial elétrico:
$$ \Rightarrow \nabla^2V = 0 \text{ (Equação de Laplace)} $$

Com isso, para modelar o comportamento de $ V(x, y, z) $ analiticamente, partir-se-á da Equação
de Laplace com as Condições de Contorno fornecidas no problema:
\begin{equation}
  \label{eq:laplace}
  \centering
  \begin{cases}
    \nabla^2 V = 0 \\
    x: V(0, y, z) = 0,\, V(a, y, z) = 0 \\
    y: V(x, 0, z) = 0,\, V(x, a, z) = 0 \\
    z: V(x, y, 0) = 0,\, V(x, y, a) = V_0
  \end{cases}
\end{equation}

A partir dela, tem-se, em coordenadas cartesianas, que:
$$ \frac{\partial^2 V}{\partial x^2} + \frac{\partial^2 V}{\partial y^2} + \frac{\partial^2 V}{\partial z^2} = 0 $$

E, pelo Método da Separação de Variáveis, assume-se que, para coordenadas cartesianas:
$$ V(x, y, z) = X(x)Y(y)Z(z) $$

Portanto, a partir daí, tem-se que:
$$ \Rightarrow \frac{\partial^2 V}{\partial x^2} = Y(y)Z(z)\frac{\partial^2}{\partial x^2}X(x) $$
$$ \Rightarrow \frac{\partial^2 V}{\partial y^2} = X(x)Z(z)\frac{\partial^2}{\partial y^2}Y(y) $$
$$ \Rightarrow \frac{\partial^2 V}{\partial z^2} = X(x)Y(y)\frac{\partial^2}{\partial z^2}Z(z) $$
$$ \Rightarrow \nabla^2 V = Y(y)Z(z)\frac{\partial^2}{\partial x^2}X(x) + X(x)Z(z)\frac{\partial^2}{\partial y^2}Y(y) + X(x)Y(y)\frac{\partial^2}{\partial z^2}Z(z) = 0 $$

Agora, assumindo que $ X(x) \neq 0 $, $ Y(y) \neq 0 $ e $ Z(z) \neq 0 $, na região de interesse, pode-se dividir a equação acima por $ X(x)Y(y)Z(z) $:
$$ \Rightarrow \frac{1}{X(x)}\frac{\partial^2 X(x)}{\partial x^2} + \frac{1}{Y(y)}\frac{\partial^2 Y(y)}{\partial y^2} + \frac{1}{Z(z)}\frac{\partial^2 Z(z)}{\partial z^2} = 0 $$

Porém, a única forma dessa equação resultar em zero para todos os valores de $ X(x) $, $ Y(y) $ e $ Z(z) $ se dá quando cada uma das parcelas somadas na equação é uma constante.
Em outras palavras:
\begin{equation}
  \label{eq:laplace_constants}
  \centering
  \begin{cases}
    \frac{1}{X(x)}\frac{\partial^2 X(x)}{\partial x^2} = -K_x^2 \\
    \frac{1}{Y(y)}\frac{\partial^2 Y(y)}{\partial y^2} = -K_y^2 \\
    \frac{1}{Z(z)}\frac{\partial^2 Z(z)}{\partial z^2} = -K_z^2
  \end{cases}
  \Rightarrow K_x^2 + K_y^2 + K_z^2 = 0
\end{equation}

Essa escolha de constantes foi feita para facilitar a dedução do resto do problema, visto que as constantes podem ser complexas.

Multiplicando cada uma das equações de \ref{eq:laplace_constants} pelas suas respectivas funções dependentes apenas de uma coordenada e somando
a constante dos dois lados das equações obtém-se o seguinte sistema de Equações Diferenciais Ordinárias (EDOs):
\begin{equation}
    \label{eq:laplace_edos}
    \centering
    \begin{cases}
      \frac{d^2 X(x)}{dx^2} + X(x)K_x^2 = 0 \\
      \frac{d^2 Y(y)}{dy^2} + Y(y)K_y^2 = 0 \\
      \frac{d^2 Z(z)}{dz^2} + Z(z)K_z^2 = 0 \\
      K_x^2 + K_y^2 + K_z^2 = 0
    \end{cases}
  \end{equation}

Dadas essas EDOs, para alguma das variáveis, pode-se obter as seguintes soluções gerais:

\vspace*{-1.5\baselineskip}
\begin{center}
  \begin{align}
    \label{eq:laplace_solutions_degenerates}
    S(s) &= A_0s + B_0,\, K_s^2 = 0 \\
    \label{eq:laplace_solutions_real}
    S(s) &= A_1\sin(K_ss) + B_1\cos(K_ss),\, K_s^2 > 0,\, K_s \in \mathbb{R} \\
    \label{eq:laplace_solutions_imaginary}
    S(s) &= A_2\sinh(K_ss) + B_2\cosh(K_ss),\, K_s^2 < 0,\, K_s \in \mathbb{I}
  \end{align}
\end{center}

Todas essas para $ S(s) = X(x), Y(y), Z(z) $. Para cada variável $ x $, $ y $ e $ z $, a
forma da solução geral depende das Condições de Fronteira.

Alguns PVCs em Eletrostática apresentam dependência em apenas algumas variáveis. No entanto,
mesmo que este tenha alguma simetria em relação à $ x $ e $ y $ (fixando um certo $ x $ ou $ y $,
e fazendo $ z $ variar em função da variável restante deve resultar no mesmo comportamento de $ V(y,z) $
ou $ V(x,z) $), de forma geral, precisar-se-á resolver o problema para todas as variáveis separadamente.

\subsection{Determinando o Comportamento das Soluções}

Para descobrir qual é o comportamento de cada variável desse problema, basta analisar as Condições de Fronteira
para duas das variáveis dadas em \ref{eq:laplace} e, pela equação \ref{eq:laplace_edos}, obter e confirmar o
comportamento da variável restante.
Para a variável $ z $:
$$ z: V(x, y, 0) = 0,\, V(x, y, a) = V_0 $$

Ou seja, observa-se um comportamento de decaímento. Quanto mais afasta-se verticalmente da placa com potencial $ V_0 $,
menor será o potencial. No entanto, esse comportamento não pode ser linear, pois esse é o caso quando há apenas duas placas
paralelas, uma com potencial não nulo e a outra com potencial nulo. Esse não é o caso para este problema porque, ao decrementar
a variável $ z $, ocorre um certo amortercimento devido ao potencial nulo das placas laterais, gerando comportamento não linear
em $ z $. Isso indica, dentre as soluções gerais para as equações, que a solução nessa variável corresponde
a um decaímento exponencial descrito pela equação \ref{eq:laplace_solutions_imaginary}. Ou seja, $ K_z^2 < 0 $. Por conta
disso, sabe-se que precisa haver $ K_x^2 > 0 $ ou (inclusivo) $ K_y^2 > 0 $ para que o resto da equação \ref{eq:laplace_edos} seja satisfeito.
Nesse caso, como é um cubo com todas as distâncias iguais e com todos os potenciais iguais, exceto na tampa, percebe-se que tanto
a variável $ x $, quanto a variável $ y $ devem apresentar o mesmo comportamento. Isso também pode ser observado diretamente nas
Condições de Fronteira dessas variáveis:
$$ x: V(0, y, z) = 0,\, V(a, y, z) = 0 $$
$$ y: V(x, 0, z) = 0,\, V(x, a, z) = 0 $$

Logo, pelas constatações acima, sabe-se que $ K_x^2 > 0 $ e que $ K_y^2 > 0 $. Portanto, obtém-se as seguintes equações gerais para as
variáveis do problema:

\vspace*{-1.5\baselineskip}
\begin{center}
  \begin{align}
    \label{eq:problem_general_solution_x}
    X(x) &= A\sin(|K_x|x) + B\cos(|K_x|x),\, K_x^2 > 0,\, K_x \in \mathbb{R} \\
    \label{eq:problem_general_solution_y}
    Y(y) &= C\sin(|K_y|y) + D\cos(|K_y|y),\, K_y^2 > 0,\, K_y \in \mathbb{R} \\
    \label{eq:problem_general_solution_z}
    Z(z) &= E\sinh(|K_z|z) + F\cosh(|K_z|z),\, K_z^2 < 0,\, K_z \in \mathbb{I}
  \end{align}
\end{center}

Será confirmado se essas constatações estão efetivamente corretas através da
análise numérica na seção \ref{sec:numerical}.

Portanto, agrupando \ref{eq:problem_general_solution_x}, \ref{eq:problem_general_solution_y} e
\ref{eq:problem_general_solution_z}, obtém-se a seguinte solução geral para o problema original:
\begin{equation}
    \label{eq:problem_general_solution}
    \centering
    {\scriptstyle V(x, y, z) = [A\sin(|K_x|x) + B\cos(|K_x|x)][C\sin(|K_y|y) + D\cos(|K_y|y)][E\sinh(|K_z|z) + F\cosh(|K_z|z)]}
\end{equation}

\subsection{Cálculo dos Coeficientes das Soluções}

Com as Condições de Fronteira, serão primeiramente calculados os coeficientes mais diretos.
Ou seja, com as condições que envolvem zerar as soluções gerais:

\begin{itemize}
  \item Usando $ V(0, y, z) = 0 $ na equação \ref{eq:problem_general_solution}:
\end{itemize}
$$ \Rightarrow V(0, y, z) = [A \cdot 0 + B \cdot 1][C\sin(|K_y|y) + D\cos(|K_y|y)][E\sinh(|K_z|z) + F\cosh(|K_z|z)] = 0 $$
$$ \Rightarrow B[C\sin(|K_y|y) + D\cos(|K_y|y)][E\sinh(|K_z|z) + F\cosh(|K_z|z)] = 0 $$

Para uma multiplicação ser nula, precisa-se algum dos termos multiplicados seja nulo. Como sabe-se que
exponenciais nunca são nulas, para essa parcela ser nula, precisar-se-ia que tanto $ E = 0 $, quanto $ F = 0 $.
No entanto, isso resulta na solução trivial para a variável $ z $, algo já constatado como falso. Logo, alguma
das outras parcelas ou ambas deve ser nula:
$$ \Rightarrow B[C\sin(|K_y|y) + D\cos(|K_y|y)] = 0 $$

O mesmo raciocínio se aplica para as constantes $ C $ e $ D $, visto que as funções seno e cosseno nunca são zero
ao mesmo tempo, exigindo que, para essa parcela ser nula, precisa-se da solução trivial para $ y $, algo analisado
anteriormente como não verdadeiro. Portanto, só resta a conclusão que:
$$ \Rightarrow B = 0 $$

\begin{itemize}
  \item Usando $ V(x, 0, z) = 0 $ na equação \ref{eq:problem_general_solution} sabendo que $ B = 0 $:
\end{itemize}
$$ \Rightarrow V(x, 0, z) = A\sin(|K_x|x)[C\cdot 0 + D\cdot 1][E\sinh(|K_z|z) + F\cosh(|K_z|z)] = 0 $$
$$ \Rightarrow D\cdot A\sin(|K_x|x)[E\sinh(|K_z|z) + F\cosh(|K_z|z)] = 0 $$

Como já argumentado acima, $ E \neq 0 $ e $ F \neq 0 $:
$$ \Rightarrow D\cdot A\sin(|K_x|x) = 0 $$

Pela mesma lógica da condição anterior, para não haver solução trivial na variável $ x $, precisa-se que $ A \neq 0 $:
$$ \Rightarrow D = 0 $$

\begin{itemize}
  \item Usando $ V(x, y, 0) = 0 $ na equação \ref{eq:problem_general_solution} sabendo que $ B = 0 $ e $ D = 0 $:
\end{itemize}
$$ \Rightarrow V(x, y, 0) = A\sin(|K_x|x)C\sin(|K_y|y)[E\cdot 0 + F\cdot 1] = 0 $$
$$ \Rightarrow F\cdot A\sin(|K_x|x)C\sin(|K_y|y) = 0 $$

Como já discutido anteriormente, para não haver solução trivial nas variáveis $ x $ e $ y $, precisa-se que $ A \neq 0 $ e $ C \neq 0 $:
$$ \Rightarrow F = 0 $$

Assim, para facilitar, chamar-se-á $ A' := A\cdot C\cdot E $, ou seja:
\begin{equation}
    \label{eq:problem_part_solution}
    \centering
    V(x, y, z) = A'\sinh(|K_z|z)\sin(|K_x|x)\sin(|K_y|y)
\end{equation}

\begin{itemize}
  \item Usando $ V(a, y, z) = 0 $ na equação \ref{eq:problem_part_solution}:
\end{itemize}
$$ \Rightarrow V(a, y, z) = A'\sinh(|K_z|z)\sin(|K_x|a)\sin(|K_y|y) = 0 $$

Como a contante $ A' $ não pode ser nula e as funções seno e seno hiperbólico são não nulas para diversos valores de $ y $ e $ z $, resta que:
$$ \Rightarrow \sin(|K_x|a) = 0 $$

A função seno é periódica e apresenta valor zero quando o seu argumento vale $ i\pi $, onde $ i \in \mathbb{Z} $:
$$ \Rightarrow |K_x|a = i\pi $$
$$ |K_x| = \frac{i\pi}{a},\, i \in \mathbb{Z} $$

\begin{itemize}
  \item Usando $ V(x, a, z) = 0 $ na equação \ref{eq:problem_part_solution}:
\end{itemize}
$$ \Rightarrow V(x, a, z) = A'\sinh(|K_z|z)\sin(|K_x|x)\sin(|K_y|a) = 0 $$

Pelo mesmo raciocínio anterior, tem-se que:
$$ \Rightarrow \sin(|K_y|a) = 0 $$
$$ \Rightarrow |K_y|a = j\pi $$
$$ |K_y| = \frac{j\pi}{a},\, j \in \mathbb{Z} $$

Agora, tomando a última equação de \ref{eq:laplace_edos}:
$$ K_x^2 + K_y^2 + K_z^2 = 0 $$
$$ \Rightarrow -K_z^2 = K_x^2 + K_y^2 $$

Como $ K_z^2 < 0 $, $ K_y^2 > 0 $ e $ K_x^2 > 0 $,
$$ \Rightarrow |K_z|^2 = |K_x|^2 + |K_y|^2 $$
$$ \Rightarrow |K_z| = \sqrt{|K_x|^2 + |K_y|^2} $$

Substituindo os valores de $ |K_x| $ e $ |K_y| $ na equação anterior, tem-se que:
$$ \Rightarrow |K_z| = \sqrt{\left(\frac{i\pi}{a}\right)^2 + \left(\frac{j\pi}{a}\right)^2} $$
$$ \Rightarrow |K_z| = \frac{\pi}{a}\sqrt{i^2 + j^2} $$

Portanto, em resumo:

\vspace*{-1.5\baselineskip}
\begin{center}
  \begin{align}
    \label{eq:Kx_values}
    |K_x| &= \frac{i\pi}{a},\, i \in \mathbb{Z} \\
    \label{eq:Ky_values}
    |K_y| &= \frac{j\pi}{a},\, j \in \mathbb{Z} \\
    \label{eq:Kz_values}
    |K_z| &= \frac{\pi}{a}\sqrt{i^2 + j^2},\, i, j \in \mathbb{Z}
  \end{align}
\end{center}

Por fim, será utilizada a condição de fronteira $ V(x, y, a) = V_0 $ em \ref{eq:problem_part_solution}.
No entanto, essa condição não é trivial de ser aplicada, visto que gera-se
a seguinte sequência de afirmações:
$$ \Rightarrow V(x, y, a) = A'\sinh(|K_z|a)\sin(|K_x|x)\sin(|K_y|y) = V_0 $$

Coletando as constantes e definindo $ C_{ij} := A'\sinh(|K_z|a) $, obtém-se que

$$ \Rightarrow C_{ij}\sin(|K_x|x)\sin(|K_y|y) = V_0 $$
$$ \Rightarrow C_{ij}\sin\left(\frac{i\pi x}{a}\right)\sin\left(\frac{j\pi y}{a}\right) = V_0 $$
\clearpage
A multiplicação de duas funções periódicas dessa forma nunca será constante. Portanto, será
preciso extrapolar o problema e considerar que a função potencial é uma função períodica ímpar tanto em
$ x $, quanto em $ y $, formando uma espécie tabuleiro de xadrez com largura de posição $ a $ no espaço:

\begin{figure}[!h]
    \centering
    \tdplotsetmaincoords{0}{130}
    \begin{tikzpicture}[tdplot_main_coords, scale=0.8]
        % Define the cube's dimensions
        \def\a{2} % length of the cube's sides
        \def\d{5} % length of the axis

        % Draw the axis
        \draw[thick,->] (-0.1\d,0,0)-- (\d,0,0) node[anchor=north east]{$x$};
        \draw[thick,->] (0,-0.1\d,0)-- (0,\d,0) node[anchor=north west]{$y$};

        % Draw the checkboard
        \filldraw[fill=black!30, draw=black] (0,0,\a) -- (\a,0,\a) -- (\a,\a,\a) -- (0,\a,\a) -- cycle;
        \filldraw[fill=black!60, draw=black] (\a,0,-\a) -- (2*\a,0,-\a) -- (2*\a,\a,-\a) -- (\a,\a,-\a) -- cycle;
        \filldraw[fill=black!60, draw=black] (0,\a,-\a) -- (\a,\a,-\a) -- (\a,2*\a,-\a) -- (0,2*\a,-\a) -- cycle;
        \filldraw[fill=black!30, draw=black] (\a,\a,\a) -- (2*\a,\a,\a) -- (2*\a,2*\a,\a) -- (\a,2*\a,\a) -- cycle;

        % Label the cube's faces and points on the axis
        \node at (0.5*\a,0.5*\a,\a) {$ +V_0 $};
        \node at (1.5*\a,0.5*\a,-\a) {$ -V_0 $};
        \node at (1.5*\a,1.5*\a,\a) {$ +V_0 $};
        \node at (0.5*\a,1.5*\a,-\a) {$ -V_0 $};

        \node at (\a,-0.25,0) {$ a $};
        \node at (-0.25,\a,0) {$ a $};
        \node at (-0.25,0.25,0) {$ a $};
        \node at (2*\a, -0.35, 0) {$ 2a $};
        \node at (-0.35, 2*\a, 0) {$ 2a $};
    \end{tikzpicture}
    \caption{\label{plot:potencial_ext} Extrapolação do Potencial além de $ x = y = z = a $.}
\end{figure}

Com essa suposição, já que o potencial fora do cubo não é importante para o problema, pode-se
aplicar a teoria das Séries de Fourrier para achar uma solução que satisfaça às condições acima.

Como há uma infinidade de múltiplos de $ i $ e $ j $ que fazem as contantes $ C_{ij} $, $ |K_x| $ e $ |K_y| $ satisfazerem as
Condições de Contorno, sabe-se que, nessa última Condição de Contorno, precisa-se de uma resposta da forma:
\begin{equation}
  \label{eq:last_condition}
  \centering
  \sum_{i=1}^{\infty} \sum_{j=1}^{\infty} C_{ij}\sin\left(\frac{i\pi x}{a}\right)\sin\left(\frac{j\pi y}{a}\right) = V_0
\end{equation}

Como sabe-se, manipulando a sua definição, que $ A' = \frac{C_{ij}}{\sinh(|K_z|a)} = \frac{C_{ij}}{\sinh(\pi\sqrt{i^2 + j^2})} $, a resposta final é da forma:
\begin{equation}
  \label{eq:problem_part2_solution}
  \centering
  V(x, y, z) = \sum_{i=1}^{\infty} \sum_{j=1}^{\infty} \frac{C_{ij}}{\sinh(\pi\sqrt{i^2 + j^2})}\sinh\left(\frac{\pi}{a}\sqrt{i^2 + j^2}z\right)\sin\left(\frac{i\pi x}{a}\right)\sin\left(\frac{j\pi y}{a}\right)
\end{equation}

Agora, resta determinar a última constante que é $ C_{ij} $ a partir da equação \ref{eq:last_condition}.

Para fazer isso, será utilizado o "Truque de Fourrier" tanto na variável $ x $, quanto na variável $ y $. Ou seja, multiplicar-se-á a equação
\ref{eq:last_condition} por $ \sin\left(\frac{n\pi x}{a}\right)\sin\left(\frac{m\pi y}{a}\right) $, com $ n, m \in \mathbb{Z} $:
$$ \sum_{i=1}^{\infty} \sum_{j=1}^{\infty} C_{ij}\sin\left(\frac{i\pi x}{a}\right)\sin\left(\frac{j\pi y}{a}\right)\sin\left(\frac{n\pi x}{a}\right)\sin\left(\frac{m\pi y}{a}\right) = V_0\sin\left(\frac{n\pi x}{a}\right)\sin\left(\frac{m\pi y}{a}\right) $$
$$ \Rightarrow \sum_{i=1}^{\infty} \sum_{j=1}^{\infty} C_{ij}\sin\left(\frac{i\pi x}{a}\right)\sin\left(\frac{n\pi x}{a}\right)\sin\left(\frac{j\pi y}{a}\right)\sin\left(\frac{m\pi y}{a}\right) = V_0\sin\left(\frac{n\pi x}{a}\right)\sin\left(\frac{m\pi y}{a}\right) $$

Agora, utilizando a identidade trigonométrica $ \sin(a)\sin(b) = \frac{\cos(a - b) - \cos(a + b)}{2} $ no lado esquerdo, obtém desse lado:
$$ \Rightarrow \sum_{i=1}^{\infty} \sum_{j=1}^{\infty} C_{ij}\left[\frac{\cos\left(\frac{i\pi x}{a} - \frac{n\pi x}{a}\right) - \cos\left(\frac{i\pi x}{a} + \frac{n\pi x}{a}\right)}{2}\right]\left[\frac{\cos\left(\frac{j\pi y}{a} - \frac{m\pi y}{a}\right) - \cos\left(\frac{j\pi y}{a} + \frac{m\pi y}{a}\right)}{2}\right]$$
$$ \Rightarrow \frac{1}{4}\sum_{i=1}^{\infty} \sum_{j=1}^{\infty} C_{ij}\left[\cos\left(\frac{\pi x(i - n)}{a}\right) - \cos\left(\frac{\pi x(i + n)}{a}\right)\right]\left[\cos\left(\frac{\pi y(j - m)}{a}\right) - \cos\left(\frac{\pi y(j + m)}{a}\right)\right]$$

Aplicando a integral dupla $ \int_{0}^{a}dy\int_{0}^{a}dx $ no lado esquerdo:
$$ \textstyle \Rightarrow \frac{1}{4}\int_{0}^{a}\int_{0}^{a}\sum_{i=1}^{\infty} \sum_{j=1}^{\infty} C_{ij}\left[\cos\left(\frac{\pi x(i - n)}{a}\right) - \cos\left(\frac{\pi x(i + n)}{a}\right)\right]\left[\cos\left(\frac{\pi y(j - m)}{a}\right) - \cos\left(\frac{\pi y(j + m)}{a}\right)\right]dxdy $$
$$ \textstyle \Rightarrow \frac{1}{4} \sum_{i=1}^{\infty} \sum_{j=1}^{\infty} C_{ij} \int_{0}^{a} \left[\cos\left(\frac{\pi x(i - n)}{a}\right) - \cos\left(\frac{\pi x(i + n)}{a}\right)\right]dx\int_{0}^{a}\left[\cos\left(\frac{\pi y(j - m)}{a}\right) - \cos\left(\frac{\pi y(j + m)}{a}\right)\right]dy $$

Neste ponto, caso $ n \neq i $ ou (inclusivo) $ m \neq j $:
$$ \scriptstyle \Rightarrow \frac{a^2}{4\pi^2} \sum_{i=1}^{\infty} \sum_{j=1}^{\infty} C_{ij} \left[\frac{1}{i - n}\sin\left(\frac{\pi x(i - n)}{a}\right) - \frac{1}{i + n}\sin\left(\frac{\pi x(i + n)}{a}\right)\right]_{0}^{a} \left[\frac{1}{j - m}\sin\left(\frac{\pi y(j - m)}{a}\right) - \frac{1}{j + m}\sin\left(\frac{\pi y(j + m)}{a}\right)\right]_{0}^{a} $$

Avaliando a integral acima, quando $ x = 0 $ e $ y = 0 $, observa-se que a afirmação acima é nula. Caso $ x = a $ e $ y = a $, o argumento dos senos será um múltiplo inteiro qualquer de $ \pi $, zerando novamente toda a afirmação. Com isso, concluí-se que:
$$ n \neq i \lor m \neq j \rightarrow \text{ a afirmação esquerda é nula.} $$

Agora, caso $ n = i $ e $ m = j $:
$$ \Rightarrow \frac{1}{4} \sum_{i=1}^{\infty} \sum_{j=1}^{\infty} C_{ij} \left[\int_{0}^{a}dx - \int_{0}^{a}\cos\left(\frac{2i\pi x}{a}\right)dx\right]\left[\int_{0}^{a}dy - \int_{0}^{a}\cos\left(\frac{2j\pi y}{a}\right)dy\right] $$
$$ \Rightarrow \frac{1}{4} \sum_{i=1}^{\infty} \sum_{j=1}^{\infty} C_{ij} \left[x - \frac{a}{2i\pi}\sin\left(\frac{2i\pi x}{a}\right)\right]_{0}^{a}\left[y - \frac{a}{2j\pi}\sin\left(\frac{2j\pi y}{a}\right)\right]_{0}^{a} $$

Ou seja, como qualquer múltiplo inteiro de $ \pi $ torna a função seno nula, tem-se, no lado esquerdo:
\begin{equation}
  \label{eq:Cij_left}
  \centering
  \frac{a^2}{4} \sum_{i=1}^{\infty} \sum_{j=1}^{\infty} C_{ij}
\end{equation}

Agora, voltando para o lado direito da equação e aplicando a mesma integral dupla:
$$ \Rightarrow \int_{0}^{a} \int_{0}^{a} V_0\sin\left(\frac{n\pi x}{a}\right)\sin\left(\frac{m\pi y}{a}\right)dxdy $$
$$ \Rightarrow V_0 \int_{0}^{a} \sin\left(\frac{n\pi x}{a}\right)dx \int_{0}^{a} \sin\left(\frac{m\pi y}{a}\right)dy $$
$$ \Rightarrow \frac{a^2V_0}{\pi^2} \left[-\frac{1}{n}\cos\left(\frac{n\pi x}{a}\right)\right]_{0}^{a} \left[-\frac{1}{m}\cos\left(\frac{m\pi y}{a}\right)\right]_{0}^{a} $$
$$ \Rightarrow \frac{a^2V_0}{\pi^2} \left[\frac{1}{n} - \frac{1}{n}\cos\left(n\pi\right)\right] \left[\frac{1}{m} - \frac{1}{m}\cos\left(m\pi\right)\right] $$

Nesse ponto, o valor da integral depende da paridade de $ n $ e $ m $. Caso $ n $ ou (inclusivo) $ m $ for par, a integral se torna nula. Logo, precisa-se que $ n = i = 2l - 1 $ e $ m = j = 2k - 1 $, com $ l, k \in \mathbb{Z} $ para que o lado
direito não seja nulo e valha:
\begin{equation}
  \label{eq:Cij_right}
  \centering
  \frac{4a^2V_0}{\pi^2 nm},\, n = 2l - 1,\, m = 2k - 1,\, l, k \in \mathbb{Z}
\end{equation}

Portanto, reunindo novamente o lado esquerdo \ref{eq:Cij_left} e o lado direto \ref{eq:Cij_right}, mantendo as restrições impostas por cada lado, tem-se:
$$ \Rightarrow \frac{a^2}{4} \sum_{l=1}^{\infty} \sum_{k=1}^{\infty} C_{(2l - 1)(2k - 1)} = \frac{4a^2V_0}{\pi^2 (2l - 1)(2k - 1)} $$
$$ \Rightarrow \sum_{l=1}^{\infty} \sum_{k=1}^{\infty} C_{(2l - 1)(2k - 1)} = \frac{16V_0}{\pi^2(2l - 1)(2k - 1)} $$

Sem perda de generalidade, como, para cada valor de iterador $ l $ e $ k $, obtém-se uma nova igualdade, obtém-se:
$$ \Rightarrow \sum_{l=1}^{\infty} \sum_{k=1}^{\infty} C_{(2l - 1)(2k - 1)} = \sum_{l=1}^{\infty} \sum_{k=1}^{\infty} \frac{16V_0}{\pi^2(2l - 1)(2k - 1)} $$
$$ \Rightarrow C_{(2l - 1)(2k - 1)} = \frac{16V_0}{\pi^2(2l - 1)(2k - 1)} $$

Retornando essa constante para a equação \ref{eq:problem_part2_solution}, tem-se que:
$$ \scriptstyle V(x, y, z) = \sum_{l=1}^{\infty} \sum_{k=1}^{\infty} \frac{16V_0}{\pi^2(2l - 1)(2k - 1)\sinh(\pi\sqrt{(2l - 1)^2 + (2k - 1)^2})}\sinh\left(\frac{\pi}{a}\sqrt{(2l - 1)^2 + (2k - 1)^2}z\right)\sin\left(\frac{(2l - 1)\pi x}{a}\right)\sin\left(\frac{(2k - 1)\pi y}{a}\right) $$

\subsection{Forma Análitica Final}
Para finalizar, reunindo a última equação anterior, movendo alguns termos para organizá-la e renomeando os iteradores, obteve-se a seguinte expressão:
\begin{equation}
  \label{eq:problem_solution_final}
  \centering
  V(x, y, z) = \frac{16V_0}{\pi^2}\sum_{i=1}^{\infty} \sum_{j=1}^{\infty} \frac{1}{ij\sinh(\pi\sqrt{i^2 + j^2})}\sinh\left(\frac{\pi}{a}\sqrt{i^2 + j^2}z\right)\sin\left(\frac{i\pi x}{a}\right)\sin\left(\frac{j\pi y}{a}\right)
\end{equation}

Onde, na expressão acima, $ i = 1, 3, 5, \ldots, 2l - 1, \ldots $ e $ j = 1, 3, 5, \ldots, 2k - 1, \ldots $

\begin{center}
  \[ \scalebox{3}{$ * $} \]
\end{center}

\section{Resolução Numérica do Problema}
\label{sec:numerical}
\subsection{Comportamento Numérico da Solução}
\subsection{Simulação em \textit{Software}}

\section{Exemplos}

\begin{itemize}
  \item $ N_1 = 3$;
  \item $ N_2 = 4$;
  \item $ N_3 = 1$;
  \item $ N_4 = 8$;
  \item $ N_5 = 1$;
  \item $ N_6 = 0$.
\end{itemize}

\begin{center}
  \[ \scalebox{3}{$ * $} \]
\end{center}

\begin{equation}
  \label{eq:quad_cascade}
  \centering
  \begin{matrix}
    a_{11} = a^{'}_{11}a^{''}_{11} + a^{'}_{12}a^{''}_{21} & \hspace{10pt} a_{12} = a^{'}_{11}a^{''}_{12} + a^{'}_{12}a^{''}_{22} \\\\
    a_{21} = a^{'}_{21}a^{''}_{11} + a^{'}_{22}a^{''}_{21} & \hspace{10pt} a_{22} = a^{'}_{21}a^{''}_{12} + a^{'}_{22}a^{''}_{22}
  \end{matrix}
\end{equation}

\end{document}